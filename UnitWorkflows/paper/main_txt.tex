\documentclass{article}
\title{Decision Tree Modeling of Frog Species}
\author{Alex Hey}
\date{\today}

\usepackage{Sweave}
\begin{document}
\input{main_txt-concordance}
\maketitle

\section{Introduction}
In this toy analysis, we applied various machine learning models to classify frog species based on MFCC data. This report presents the results of decision tree models trained on a subset of the data and evaluates their performance.

\section{Methods}
\subsection{Data Preparation}
The dataset, \texttt{Frogs\_MFCCs.csv}, was split into a training set and a test set with a 75-25 ratio. 

\subsection{Model Training}
We trained decision trees with and without pruning. A k-fold cross-validation was performed using the \texttt{caret} package to assess model performance.

\section{Results}
\subsection{Error Rates}
The error rates for each model variant are summarized in Table~\ref{tab:results}. Based on these results, we observed that the unpruned tree performed slightly better than the pruned tree.

%DAN: results="asis" did not work for me. 
